\documentclass[11pt]{article} 



\makeatletter
\renewcommand\section{\@startsection{section}{1}{\z@}%
                                  {-3.5ex \@plus -1ex \@minus -.2ex}%
                                  {2.3ex \@plus.2ex}%
                                  {\normalfont\large\bfseries}}
\makeatother

\addtolength{\oddsidemargin}{-.875in}
\addtolength{\evensidemargin}{-.875in}
\addtolength{\textwidth}{1.75in}
\addtolength{\topmargin}{-.875in}
\addtolength{\textheight}{1.75in}


\usepackage{amssymb}
\usepackage{amsmath}
\usepackage{amsthm}
\usepackage{graphicx,caption,subcaption}


\usepackage{titling}
\setlength{\droptitle}{-6em}
\posttitle{\par\end{center}\vspace{-4.8em}}

\newcommand{\R}{{\ensuremath{\mathbb{R}}} }
\newcommand{\Q}{{\ensuremath{\mathbb{Q}}} }
\newcommand{\C}{{\ensuremath{\mathbb{C}}} }
\newcommand{\N}{{\ensuremath{\mathbb{N}}} }
\newcommand{\Z}{{\ensuremath{\mathbb{Z}}} }

\newcommand{\sexion}{\addtocounter{section}{1} }


\newcommand{\hint}[1]{{(\emph{Hint:} #1)}} %This line shows hints
%\newcommand{\hint}[1]{} %Use this line to hide all hints


\usepackage{fancyhdr}
\pagestyle{fancyplain}
\renewcommand{\headrulewidth}{0pt}

\begin{document}

%\lhead{}
\rhead{Matrix Exponentiation}
\section{Problem}
Find $e^{A t}$ for
\[A = \left( \begin{array}{cc} 5 & -4 \\ 8 & -7 \end{array} \right)\]

\section{Solution $1$}

First we find the eigenvectors $ \lambda_1 = 1, \lambda_2 = -3$ and their corresponding eigenvectors $v_1 =  (1~1)^T, v_2 = (1~2)^T$. With these in hand we can diagonalize the matrix:
\[ \left( \begin{array}{cc} 5 & -4 \\ 8 & -7 \end{array} \right)  = \underbrace{ \left( \begin{array}{cc} 1 & 1 \\ 1 & 2 \end{array} \right)}_B \underbrace{ \left( \begin{array}{cc} 1& 0 \\ 0 & -3 \end{array} \right)}_D \underbrace{ \left( \begin{array}{cc} 2 & -1 \\ -1  & 1 \end{array} \right)}_{B^{-1}}  \]
The matrix on the left comes from the eigenvectors, the one on the right is inverse of the one on the left and the middle one has the eigenvalues on the diagonal.

Let's recall a  fact about matrices. If $D$ is a diagonal matrix we have
\[( P D P^{-1} ) ^n =\underbrace{ P D P^{-1} \cdot P D P^{-1} \cdots P D P^{-1}}_{n\mathrm{\ times}} = PD^n P^{-1}\]


Now remember the formula for matrix exponential is
\[e^{At} = I  + A + \frac{1}{2} A^2 t^2 + \frac{1}{6} A^3 t^3 + \cdots\]
so
\begin{align*}
e^{At} &= I  + A + \frac{1}{2} A^2 t^2 + \frac{1}{6} A^3 t^3 + \cdots \\
&= I  + BDB^{-1} + \frac{1}{2} (BDB^{-1})^2 t^2 + \frac{1}{6} (BDB^{-1})^3 t^3 + \cdots\\
&= I  + BDB^{-1} + \frac{1}{2} BD^2B^{-1} t^2 + \frac{1}{6} BD^3B^{-1} t^3 + \cdots\\
&=  B \left(I + D + \frac{1}{2} D^2 t^2 + \frac{1}{6} D^3 t^3 + \cdots \right) B^{-1}\\
&=  B e^{Dt} B^{-1}\\
\end{align*}
But exponentiating diagonal matrices is easy!
\begin{align*}
e^{At} &= B e^{Dt} B^{-1} \\
&=    \left( \begin{array}{cc} 1 & 1 \\ 1 & 2 \end{array} \right)   \left( \begin{array}{cc} e^t & 0 \\ 0 & e^{-3t} \end{array} \right) \left( \begin{array}{cc} 2 & -1 \\ -1  & 1 \end{array} \right)\\
&= \left(
\begin{array}{cc}
 -e^{-3 t}+2 e^t & e^{-3 t}-e^t \\
 -2 e^{-3 t}+2 e^t & 2 e^{-3 t}-e^t \\
\end{array}
\right)
\end{align*}
 



\end{document}

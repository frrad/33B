\documentclass[14pt]{article}

\usepackage{amsthm, amsmath, graphicx} 

\title{33B: Notes}
\author{Frederick Robinson}

\newtheorem{defn}{Definition}
\newtheorem{re}{Recipe}
\newtheorem{ex}{Example}

\newcommand{\type}[1]{\begin{center} \emph{\textbf{#1}} \end{center}}
\newcommand{\exs}{\subsubsection*{Examples}}



\begin{document}

\maketitle
\tableofcontents
\pagebreak

Frederick Robinson

frobinson@math.ucla.edu

math.ucla.edu/$\sim$frobinson

Office Hours: ----- (or by appointment)
MS 3969

\section*{Format}

Generally I'll spend the first part (at most $1/2$) of class on exposition, recapping the material
from class. The rest of class will be spent on working examples similar to those on your homework. 

\pagebreak

\section{Introduction}

Differential equations are just equations involving the derivatives of functions. There are partial
differential equations - differential equations involving partial derivatives of functions, and
ordinary differential equations - differential equations which just involve ordinary derivatives. 

We can also classify differential equations by the order of the derivatives they involve:
\begin{defn}[Order]The \emph{order} of a differential equation is just the order of the highest
  order derivative that it involves.\end{defn}

Differential equations don't specify just one solution, but rather a family of solutions. For
instance consider $x' = 3$. This equation is solved by $x = 3t$ and $x = 3t + 5$. In general
solutions are exactly those equations of the form $x = 3t + C$. We call this the \emph{general
  solution}.

\begin{defn}[General Solution]
  The \emph{general solution} to an ordinary differential equation of order $n$ is an equation
  containing $n$ constants which describes all possible solutions to the equation. 
\end{defn}

Often we're interested in one particular solution to a differential equation which satisfies
constraints of the form $y^{(n)}(t) = y_0$. We call these constraints \emph{initial
  conditions}. This sort of problem is called an \emph{initial value problem}. It takes $n$
conditions to fully specify a \emph{particular solution} to a differential equation of order $n$.

Suppose in our example above that in addition to the differential equation $x'=3$ we had specified
that $x(5) = 4$, then we could find a particular solution by first finding the general solution $x =
3t + C$, and then plugging in the initial condition: $x(5) = 3 \cdot 5 + C = 4 \Rightarrow C =
-1$. Our solution is then just $x(t) = 3 t -1 $.


\subsection*{Examples}
Give some examples of PDE / ODEs of different orders...

Differential equations come up all the time in practical applications. Here's a really simple
example problem. 

\begin{ex}
  A ball is dropped from a height of 100m. It's position as a function of time satisfies the (second
  order linear) differential equation
  \[\frac{d^2x}{dt^2} = -10 m/s^2 \]
  When does it hit the ground?
\end{ex}

We should be able to solve this without any new techniques. Just integrate.
  \[\frac{d^2x}{dt^2} = -10 m/s^2 \Rightarrow \frac{dx}{dt} = -10 t + C \Rightarrow x(t) = -5 t^2 +
  C t + D \] Of course, this doesn't really specify the location of our ball. We need to use our
  initial conditions. Since we dropped ball, initial acceleration is $x'(0) = 0$, also we are given
  $x(0) = 100$. Substituting these into general solution we have:
  \[x'(0) = -10 t + C = C = 0\] and
  \[x(0) = -5 t^2 + C t + D = D = 100.\]

Thus, solve
\[0 = x(t) = -5 t^2 + 100 \Rightarrow t = \pm \sqrt{20}.\]
Clearly our solution must be positive, so the answer is $2 \sqrt 5$ seconds.

Of course this is a sort of silly example since we can solve it by integration. We'll solve harder
problems later using more sophisticated techniques.

\section{Chapter 2}

\subsection{Section 1}

If we have a purported solution $y(t)$ to some differential equation, we can check that it's
actually valid by differentiating and checking to see that the derivatives satisfy whatever equation
they are supposed to satisfy.

\begin{defn}[Interval of Existence]
  The \emph{interval of existence} for an initial value problem is the largest interval on which a
  solution exists and satisfies the differential equation.
\end{defn}

For instance, the interval of existence for $x' =3; x(0)=0$ is $(-\infty, \infty)$ since the
solution $x = 3t$ exists, and satisfies the given differential equation for all time.


\begin{defn}[Normal form]
  An order $n$ differential equation is said to be in \emph{normal form} if it is of the form $y^{(n)} = f(y^{(n-1)}, y^{(n-2)}, \dots, y',y,t)$.
\end{defn}

Given a first order ODE in normal form $y' = f(y,t)$ we can draw a vector field describing solutions
by drawing small lines of slope $f(y,t)$ at some collection of points $(y,t)$.

\exs

\type{Check}
\begin{ex}[Exercise 2.1.3] Check that $y' = -ty$ is solved by $y = C e ^{-(1/2) t^2}$\end{ex}

Just differentiate and compare:
\[y' = -t C e^{-(1/2) t^2} = -t y\]
so it is a solution to the differential equation.

\type{Interval of Existence}
\begin{ex}[Exercise 2.1.13] Find the interval of existence for the differential equation $y' =
  \frac{2}{3} t - \frac{5}{3 t^2}  $ satisfying initial condition $y(1) = 2$ \end{ex}

We can solve by integration:
\begin{align*}
\frac{dy}{dt}  &= \frac{2}{3} t - \frac{5}{3 t^2}\\
dy &= \left( \frac{2}{3} t - \frac{5}{3 t^2} \right) dt\\
y &= \frac{1}{3} t^2 + \frac{5}{3t} + C
\end{align*}
This is the general solution. If we further demand $y(1) = 2$
\[y(1) = \frac{1}{3} + \frac{5}{3} + C = 2 \Rightarrow C = 0\]
Therefore
\[y(t) = \frac{1}{3} t^2 + \frac{5}{3 t} \]
is the particular solution we're after.

Now that we have the solution it's easy to determine interval of existence. There's an asymptote at
$t=0$, so that is the lower end of the interval. The function is continuous as $t \to \infty$, so
there is no upper limit. Interval of existence is therefore $(0,\infty)$.

\type{Normal Form}
\begin{ex}[Exercise 2.1.1] Put $\phi(t,y,y') = t^2 y' + (1+t) y = 0$ in normal form. \end{ex}
Just solve for $y'$: 
\[y' = - \frac{y(1+t)}{t^2} \]


\type{Vector Field}
\begin{samepage}
\begin{ex}[Exercise 2.1.17] Draw vector field for $y' = y+ t$. \end{ex}
\[\includegraphics[scale=.7]{figures/{2.1.17}.pdf}\]
\end{samepage}

\begin{samepage}
\begin{ex}[Exercise 2.1.19] Draw vector field for $y' = t \tan{y/2}$. \end{ex}
\[\includegraphics[scale=.7]{figures/{2.1.19}.pdf}\]
\end{samepage}

\pagebreak

\begin{samepage}
\begin{ex}[Exercise 2.1.19] Draw vector field for $y' = \frac{x+y}{x-y}$. \end{ex}
\[\includegraphics[scale=.7]{figures/{2.1.extra}.pdf}\]
\end{samepage}

\subsection{Section 2 - Separation of Variables}

\exs

\begin{ex}[2.2.5] Find the general solution for $y' = y(x+1)$. \end{ex}

\[y = C e^{1/2 x^ 2 + x }\]

\begin{ex}[2.2.13] Find the exact solution to the IVP. Indicate interval of existence $ \frac{dy}{dx} = y/x$; $y(1) = -2$. \end{ex}

$y = -2x$, $x \in (0, \infty)$ since DE undefined at $x = 0$.


\begin{ex}[2.2.15] Find the exact solution to the IVP. Indicate interval of existence $ \frac{dy}{dx} = \frac{\sin x}{y}$; $y(\pi / 2 ) = 1$.\end{ex}

\begin{align*}
ydy &= \sin x dx \\
1/2 y^2 &= -\cos x + C \\
y^2 &= -2 \cos x + C \\
y &= \pm \sqrt{-2 \cos x + C } \\ 
\end{align*}

initial condition means that 
\[y(\pi /2 )  = 1 = \sqrt{-2 \cos \pi / 2 + C}\]
so $C =1$ and the solution is 
\[y = \sqrt{1 - 2 \cos x}.\]
The interval of existence is 
\[2 \cos x < 1 \Leftrightarrow \pi / 3 < x < 5 \pi /3 .\]
There is no equality because original equation is undefined.

\subsection{Section 4 - Linear Equations}

\begin{defn}[Linear Homogeneous] A \emph{linear homogeneous} differential equation is one of the form \[x'(t) = a(t) x(t) \] for some function $a(t)$. \end{defn}

We can solve these by separation of variables:
\begin{align*} 
\frac{1}{x} dx &= a(t) dt \\ 
\ln x &= \int a(t) dt + C \\
x &= C e^{\int a(t) dt}
\end{align*}

\begin{defn}[Linear Inhomogeneous] A \emph{linear inhomogeneous} differential equation is one of the form \[x'(t) = f(t) x(t) + g(t) \] for some functions $f(t), g(t)$. \end{defn}

We have two main techniques for solving these: Integrating Factor, and Variation of Parameter.

\begin{re} Integrating Factor:
\begin{enumerate} 
\item Write $x' - ax = f$.
\item Multiply by $u(t) = e^{- \int a(t) dt}$ to get $ (u x)' = uf $.
\item Integrate to get $u(t) x(t) = \int u(t) f(t) dt + C $.
\item Solve for $x$. 
\end{enumerate}
\end{re}

\begin{re} Variation of Parameter:
\begin{enumerate}
\item Put $y' = a y + f $, and solve  associated homogeneous equation $ y'_{hom} = a y_{hom}$.
\item Substitute guess $y = v(t) y_{hom}$ into original equation and solve for $v$. 
\item Write general solution $y = v(t) y_{hom}$.
\end{enumerate}
\end{re}

\exs

\type{Variation of Parameter}

\begin{ex} Solve $y' -2 y = t^2 e^{2 t  }$ by variation of paramenter.\end{ex}

First solve the associated homogeneous: $y'_{hom} = 2 y_{hom}$.
\begin{align*} \frac{1}{y} dy &= 2 dt \\
ln y &= 2 t + C \\
y_{hom} &= C e^{2t}
\end{align*}

Now we `guess' $ y = v(t) y_{hom}$. Since $y' = v' y_{hom} + v y'_{hom} = v' e^{2t} + 2  v e^{2t}$ we have:
\[v' e^{2t} + 2  v e^{2t} - 2(v e^{2t}) = t^2 e^{2t} \Rightarrow v' = t^2 \Rightarrow v = \frac{1}{3} t^3 + C\]
Therefore, $y = e^{2t} \left( \frac{1}{3} t^3 + C \right)$.

\begin{ex}\label{vopint} Solve  $y' + y / t = 3 \cos{(2t)}$ by variation of parameter.\end{ex}

First solve $y_{hom}  + y_{hom} / t = 0$
\begin{align*} \int \frac{1}{y} dy &= - \int \frac{1}{t} dt \\ y_{hom} &= C / t\end{align*}
So put $y = v(t) / t$. Compute $y ' = v' y + v y' = v' / t - v / t^2$. Substituting back in to original equation gives us:
\[v' / t - v / t^2 + v / t^2 = 3 \cos{(2t)} \Rightarrow v' = 3t \cos{(2t)}\]
We can compute $\int t \cos{(2t)}$ by parts. Taking $u = t dv = \cos 2t, du = 1, v = 1/2 \sin 2t$ gives us
\[\int t \cos 2 t = \frac{1}{2} t \sin 2t  - \frac{1}{2} \int \sin 2t= \frac{t}{2}  \sin 2t  + \frac{1}{4}\cos{(2t)}.\]
Thus, $v =  \frac{3 t}{2}  \sin 2t  + \frac{3}{4}\cos{(2t)}+ C $ and the answer is
\[y = \frac{3 }{2}  \sin 2t  + \frac{3}{4t }\cos{(2t)}+ C /t  \]

\type{Integrating Factor}

\begin{ex} Solve $y' -2 y = t^2 e^{2 t  }$ by integrating factor.\end{ex}

Since this is already in the `right' form, we can immediately compute the integrating factor
\[u = e^{-\int2 dt } = e^{-2t}.\]

Therefore, 
\[ u x =\int uf + C  = \int t^2 e^{2t} e^{-2t} + c = \frac{1}{3} t^3 + C \Rightarrow x = e^{2t} \left( \frac{1}{3} t^3 + C \right)\]

\begin{ex}Solve  $y' + y / t = 3 \cos{(2t)}$ by integrating factor.\end{ex}

Compute the integrating factor $a = e^{\int 1/ t\ dt } = t$ and then write
\begin{align*} u x &= \int u f + C \\ t x &=  \int 3t \cos{(2t)} + C\end{align*} 
integrating by parts as in example \ref{vopint} gives us
\[x = \frac{3 }{2}  \sin 2t  + \frac{3}{4t }\cos{(2t)}+ \frac{C}{t}  \]



\subsection{Section 5 - Mixing Problems}

Mixing problems are a class of examples of differential equations involving mixing liquids. We assume `perfect mixing.' The most important equation is
\begin{align}\label{mixing} \frac{dx}{dt} = \mbox{Rate In} - \mbox{Rate Out} \end{align}
It's often helpful to look at the units and use `dimensional analysis' as sanity check.

\exs


\begin{ex}
A 50-gallon tank initially contains 20 gallons of pure water. Salt water solution with concentration of 1/2 lb/gal is added at a rate of 4 gal/min. A drain allows salt water to leave at 2 gal/min. How much salt is in the tank when it fills?
\end{ex}

Use equation \ref{mixing} (note: units should be lb/gal):
\[\frac{dx}{dt} = \mbox{Rate In} - \mbox{Rate Out}\]

First Rate in:
\[RI = \frac{1}{2} \frac{lb}{gal} \cdot 4 \frac{gal}{min} = 2 \frac{lb}{min}\]

For rate out we have:
\[RO = 2 \frac{gal}{min } \cdot \frac{x(t)}{v(t)} \frac{lb}{gal}\]
where $v(t) = 20 + 2t$ gives the volume of water in the tank as a function of time.

All together we have
\[\frac{dx}{dt} = 2 - \frac{x}{10 + t } \Leftrightarrow x' + \frac{1}{10 + t} x = 2\]
We can solve by integrating factor:
\[u = e^{\int \frac{1}{10 + t } dt } = 10 + t\]
\[(10 + t ) x = 2 \int (10 + t ) dt + C \Rightarrow x = \frac{t^2 + 20 t + C}{t + 10}\]

Using initial value $x(0) = 0$ (since we started with pure water) we get $C = 0$, and 
\[  x =   \frac{t(t  + 20)  }{t + 10}.\]
To find when the tank fills solve $50 = v(t) = 20  + 2 t  \Rightarrow t =15$. Substituting gives us $x(15) = 21$.

\end{document}

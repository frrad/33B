\documentclass[14pt]{article}

\usepackage{amsthm, amsmath, graphicx} 

\title{33B: Notes}
\author{Frederick Robinson}

\newtheorem{defn}{Definition}
\newtheorem{ex}{Example}

\begin{document}

\maketitle

\pagebreak

Frederick Robinson

frobinson@math.ucla.edu

math.ucla.edu/$\sim$frobinson

Office Hours: ----- (or by appointment)
MS 3969

\section*{Format}

Generally I'll spend the first part (at most $1/2$) of class on exposition, recapping the material
from class. The rest of class will be spent on working examples similar to those on your homework. 

\pagebreak

\section{Introduction}

Differential equations are just equations involving the derivatives of functions. There are partial
differential equations - differential equations involving partial derivatives of functions, and
ordinary differential equations - differential equations which just involve ordinary derivatives. 

We can also classify differential equations by the order of the derivatives they involve:
\begin{defn}[Order]The \emph{order} of a differential equation is just the order of the highest
  order derivative that it involves.\end{defn}

Differential equations don't specify just one solution, but rather a family of solutions. For
instance consider $x' = 3$. This equation is solved by $x = 3t$ and $x = 3t + 5$. In general
solutions are exactly those equations of the form $x = 3t + C$. We call this the \emph{general
  solution}.

\begin{defn}[General Solution]
  The \emph{general solution} to an ordinary differential equation of order $n$ is an equation
  containing $n$ constants which describes all possible solutions to the equation. 
\end{defn}

Often we're interested in one particular solution to a differential equation which satisfies
constraints of the form $y^{(n)}(t) = y_0$. We call these constraints \emph{initial
  conditions}. This sort of problem is called an \emph{initial value problem}. It takes $n$
conditions to fully specify a particular solution to a differential equation of order $n$.

Suppose in our example above that in addition to the differential equation $x'=3$ we had specified
that $x(5) = 4$, then we could find a particular solution by first finding the general solution $x =
3t + C$, and then plugging in the initial condition: $x(5) = 3 \cdot 5 + C = 4 \Rightarrow C =
-1$. Our solution is then just $x(t) = 3 t -1 $.


\subsection*{Examples}
Give some examples of PDE / ODEs of different orders...

Differential equations come up all the time in practical applications. Here's a really simple
example problem. 

\begin{ex}
  A ball is dropped from a height of 100m. It's position as a function of time satisfies the (second
  order linear) differential equation
  \[\frac{d^2x}{dt^2} = -10 m/s^2 \]
  When does it hit the ground?
\end{ex}

We should be able to solve this without any new techniques. Just integrate.
  \[\frac{d^2x}{dt^2} = -10 m/s^2 \Rightarrow \frac{dx}{dt} = -10 t + C \Rightarrow x(t) = -5 t^2 +
  C t + D \] Of course, this doesn't really specify the location of our ball. We need to use our
  initial conditions. Since we dropped ball, initial acceleration is $x'(0) = 0$, also we are given
  $x(0) = 100$. Substituting these into general solution we have:
  \[x'(0) = -10 t + C = C = 0\] and
  \[x(0) = -5 t^2 + C t + D = D = 100.\]

Thus, solve
\[0 = x(t) = -5 t^2 + 100 \Rightarrow t = \pm \sqrt{20}.\]
Clearly our solution must be positive, so the answer is $2 \sqrt 5$ seconds.

Of course this is a sort of silly example since we can solve it by integration. We'll solve harder
problems later using more sophisticated techniques.

\section{Chapter 2}

\subsection{Section 1}

If we have a purported solution $y(t)$ to some differential equation, we can check that it's
actually valid by differentiating and checking to see that the derivatives satisfy whatever equation
they are supposed to satisfy.

\begin{defn}[Interval of Existence]
  The \emph{interval of existence} for an initial value problem is the largest interval on which a
  solution exists and satisfies the differential equation.
\end{defn}

For instance, the interval of existence for $x' =3; x(0)=0$ is $(-\infty, \infty)$ since the
solution $x = 3t$ exists, and satisfies the given differential equation for all time.


\begin{defn}[Normal form]
  A differential equation is said to be in \emph{normal form} if it is of the form $y' = f(y,t)$.
\end{defn}

Given a first order ODE in normal form $y' = f(y,t)$ we can draw a vector field describing solutions
by drawing small lines of slope $f(y,t)$ at some collection of points $(y,t)$.

\subsubsection*{Examples}

\emph{Check}
\begin{ex}[Exercise 2.1.3] Check that $y' = -ty$ is solved by $y = C e ^{-(1/2) t^2}$\end{ex}

Just differentiate and compare:
\[y' = -t C e^{-(1/2) t^2} = -t y\]
so it is a solution to the differential equation.

\emph{Interval of Existence}
\begin{ex}[Exercise 2.1.13] Find the interval of existence for the differential equation $y' =
  \frac{2}{3} t - \frac{5}{3 t^2}  $ satisfying initial condition $y(1) = 2$ \end{ex}

We can solve by integration:
\begin{align*}
\frac{dy}{dt}  &= \frac{2}{3} t - \frac{5}{3 t^2}\\
dy &= \left( \frac{2}{3} t - \frac{5}{3 t^2} \right) dt\\
y &= \frac{1}{3} t^2 + \frac{5}{3t} + C
\end{align*}
This is the general solution. If we further demand $y(1) = 2$
\[y(1) = \frac{1}{3} + \frac{5}{3} + C = 2 \Rightarrow C = 0\]
Therefore
\[y(t) = \frac{1}{3} t^2 + \frac{5}{3 t} \]
is the particular solution we're after.

Now that we have the solution it's easy to determine interval of existence. There's an asymptote at
$t=0$, so that is the lower end of the interval. The function is continuous as $t \to \infty$, so
there is no upper limit. Interval of existence is therefore $(0,\infty)$.

\emph{Normal Form}
\begin{ex}[Exercise 2.1.1] Put $\phi(t,y,y') = t^2 y' + (1+t) y = 0$ in normal form. \end{ex}
Just solve for $y'$: 
\[y' = - \frac{y(1+t)}{t^2} \]


\emph{Vector Field}
\begin{ex}[Exercise 2.1.17] Draw vector field for $y' = y+ t$. \end{ex}
\includegraphics{{2.1}.pdf}


\end{document}
